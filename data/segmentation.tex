%segmentation


\section{引入分割任务}

图像分割与深度图估计联系具有很强的相关系--同一幅图, 在一个分割区域内的深度应该是相近的. 在深度估计任务中引入分割还有一个比较现实的重要原因是在自动驾驶环境下, 视野内的车辆如果通过分割较好的处理后, 能有效减少非朗播面(non-Lambertian)如后车窗造成的估计误差, 另外分割还能减缓深洞效应, 这在\cite{Godard2019monodepth2}中讨论过. 另外,比较直观的是分割也能对光流有效促进, 因为刚性物体的同一个分割区域光流应该是近似相等的.

在关于深度估计的研究中,对分割任务的要求并没有那么苛刻, 其最主要的目的一般在于将视频中中的运动物体分割出来, 以此将静态背景与动态景物分开处理.\cite{Zhou2017}首次将分割引入深度估计任务, 这里其通过explaintity mask将图中的动态景物区域和遮挡区域掩住, 并将这些区域单独设计损失函数进行统计.该文中mask的预测与pose一起联合训练, 结构如\ref{fig:exp_posenet}所示

\begin{figure}[htbp]
	\centering
	\includegraphics[width=0.6\linewidth]{other/exp_pose_net.jpg}
	\caption{\cite{Zhou2017}中的exp-posenet网络结构图,pose和explainty mask一起估计,做联合训练.}
	\label{fig:exp_posenet}
\end{figure}


\begin{figure}[htbp]
	\centering
	\includegraphics[width=0.6\linewidth]{other/mask.png}
	% \vspace{\figcapmargin}
	\caption{\cite{Zhou2017}的mask示意图, 高亮部分为掩盖区域}
	\label{fig:mask}
\end{figure}


\cite{Ranjian2019cc}是首次比较全面的将四个任务联合,采用了竞争合作策略, 在来解决深度估计问题. 并证明了这种方法能使所有子问题的无监督方法的最佳性能.


\begin{figure*}[t]
	\centering
	\includegraphics[width=.8\linewidth]{other/teaser.png}
	\caption{\cite{Ranjian2019cc}的pipline. The network $R=(D,C)$ reasons about the scene by estimating optical flow over static regions using depth, $D$, and camera motion, $C$. The optical flow network $F$ estimates flow over the whole image. The motion segmentation network, $M$, masks out static scene pixels from $F$ to produce composite optical flow over the full image. A loss, $E$, using the composite flow is applied over neighboring frames to train all these models jointly. }
	\label{fig:teaser}
	\vspace{-0.3cm}
\end{figure*}