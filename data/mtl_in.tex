


\section{多任务学习中的任务关系(基于视觉)}
\label{sec:mtl_in}

多任务学习中的任务,在不同的情况下意义不同,有时是代表不同尺度或者视觉层次的相同任务\cite{kokkinos2017ubernet},有时是代表一个视觉任务的不同阶段\cite{zhang2016joint},有时也代表不同种类的视觉任务\cite{Ranjian2019cc}。
本文根据应用,将多任务分成两个分支,分别是以场景感知为主,模型多数为密集估计(Dense Prediction)的多任务集合,以及以场景监测为主,模型多数为稀疏估计的多任务集合。

\subsection{场景感知类任务集}

在场景感知类任务集中,重点处理对象为场景的背景,即全局性感知,涉及任务包括深度估计\cite{Eigen2014,Eigen2015},场景表面法向估计\cite{yin2019enforcing},边缘检测\cite{yang2018lego},光流估计\cite{fischer2015flownet,Yin2018geonet,Zou2018dfnet},语义分割\cite{Ranjian2019cc,klingner2020self},视觉里程计\cite{sfmlearner},自运动速度检估计\cite{packnet}任务, 图.\ref{fig:sfmlearner} 和图.\ref{fig:cc}分别是两个任务以及四个任务下的多任务学习框架,可以看到每个子任务通过几何关系偶合在一起,并企图提升各自任务的效果。

此类任务的一个特点就是数据难以标注,人工处理困难极大等。
例如, 光流任务是要估计出图像中每个像素的移动, 并通过色彩映射标注出来, 以此估计相对相机视角下每个点的运动状况,但此种任务无法直接标注.
此类任务的另一个特点就是耦合关系较强,各个任务之间的层次性较弱. 
文献\cite{taskonomy,standley2020tasks}主要介绍了此类任务集的耦合关系, 并量化了两两任务之间的关联度, 并且还在\cite{zamir2020robust}中基于前面的工作提出了如何鲁棒学习的策略,使得模型的泛化能力更好。

\begin{figure}[htbp]
	\begin{center}
		\includegraphics[width=1.0\linewidth]{fig/sfmlearner.pdf}
		\caption{基于深度估计任务与相机自运动位姿变换两个任务下的框图\cite{sfmlearner}}
		\label{fig:sfmlearner}%label放在caption下面貌似才得行,要不然有问
	\end{center}
\end{figure}


\begin{figure}[htbp]
	\begin{center}
		\includegraphics[width=1.0\linewidth]{fig/cc}
		\caption{包含了深度估计,位姿变换,分割以及光流估计四个任务的框图\cite{Ranjian2019cc}}
		\label{fig:cc}%label放在caption下面貌似才得行,要不然有问
	\end{center}
\end{figure}

以上任务多数都是基于视觉导航类的任务需求,重点在探测场景感知环境上,相机一般是运动状态,算法一般搭载在机器人上,计算资源有限,与场景监测类任务有着较大的不同。

\subsection{场景监测类任务集}

在场景监测类任务集中, 重点处理对象是场景中的物体,即局部性感知,涉及任务包括物体检测,跟踪,运动轨迹预测,实例分割,物体位姿估计等.
该任务集中的子任务抽象程度较高, 没有明显的关系来关联, 因此相关研究较为空白.
此类任务的特点之一就是相比于场景感知类任务的数据,标注较为容易。
第二个特点是此类任务的关键难点,也是本文研究的重点,即任务之间的耦合性并非通过几何耦合这类数学关系体现,而是通过逻辑关系,各个任务之间的层次性较强。
因此,本文根据不同的视觉任务以及层次关系,我们将该任务集绘制如图.\ref{fig:tasks}

\begin{figure}[htbp]
	\begin{center}
		\includegraphics[width=1.0\linewidth]{fig/tasks}
		\caption{非密集估计类任务层次关系}
		\label{fig:tasks}%label放在caption下面貌似才得行,要不然有问
	\end{center}
\end{figure}
在场景检测类视觉任务衍生出的应用级任务中,人脸检测以精度要求高,难度较大,应用前景广阔收到更大的关注。由于需要在大量人脸中准确分别,所以一些人脸检测应用中引入了更多辅助任务以增强性能。
Zhang等人\cite{zhang2016joint}提出通过堆叠式多重网络来进行该任务,对人脸分类、锚箱回归、面部特征点定位作为三个任务,并分别通过网络各自进行任务。
但此种方法的任务实质上是一个应用级视觉任务的不同阶段,子任务的结果无法直接拿来应用,与之前提及的任务有本质不同。
同样是人脸识别任务,Ranjan等人\cite{ranjan2017hyperface}则零任务集中包含了人脸检测、关键点定位、人脸姿态估计以及性别识别四个子任务,效果如图.\ref{fig:hyper-face}


\begin{figure}[htbp]
	\begin{center}
		\includegraphics[width=1.0\linewidth]{fig/hyperface.png}
		\caption{HyperFace效果展示,可见程序的结果包含人脸检测(框出)、脸部关键点定位(绿点)、性别识别(红蓝色框)以及姿态估计(俯仰偏转翻滚角度)。}
		\label{fig:hyper-face}%label放在caption下面貌似才得行,要不然有问
	\end{center}
\end{figure}

