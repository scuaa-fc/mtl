%motion

\section{引入运动估计任务}
\label{motion}
\paragraph{运动估计}

在引入运动估计的深度估计问题前, 先来讨论一下运动估计问题. 运动估计在本文中指的是视频中的相机自运动估计问题(Ego-motion estimation from videos), 在SLAM(Simultaneous Localization And Mapping)的前端领域中被称为视觉里程计(Visual Odometry), 是研究的比较久的一个问题\cite{gaoxiang2017}, 旨在通过相邻两帧图像计算出机器人的位姿移动, 包括位置移动和姿态转角, 从而绘制出机器人运动的完整路线图. 长久以来(直到现在)基于特征点的方法被认为是该领域中主流的方法, 它运行稳定, 对光照动态物体不敏感, 是目前比较成熟的解决方案.

得益于深度学习的发展, \cite{Kendall2016}首次提出了基于深度学习的运动估计解决办法, 能仅使用相邻两帧图像通过端到端的方法直接得到一个6维向量, 代表相机运动的6-DOF位姿移动.

本文也提到了一些相比关于传统经典方法例如SIFT等基于匹配方法的优势, 比如在运动模糊下和场景泛化的鲁棒性更好. 更重要的是可以嵌入到其他深度学习方法, 一起来进行端到端学习. 

%联合LSTM的posenet 变体阐述




\paragraph{深度估计结合运动估计方法}

\cite{Zhou2017}首次引入运动估计来作为辅助来更好的提升深度估计的性能, 并且也是第一个无监督单目训练的框架. 其整体架构使用了两个子网络,一个是深度估计网络,实用的是\cite{Mayer2016}中的dispNet作为预训练网络; 另一个是运动估计网络,使用的是\cite{Kendall2016}中的PoseNet作为预训练网络, 但两个网络在训练时没有参数共享, 仅通过损失函数将两个任务联系起来,并将两个网络的参数一起训练.
%sfmLearner架构图
\begin{figure}[htbp]
	% caption放上面就会显示在图的上方, 出现在下面就是出现在图的下方
	\begin{center}
		\includegraphics[width=\linewidth]{other/sfmlearner}
		\caption{基于视图综合的监管流程概述. 深度网络仅将目标视图作为输入, 并输出每像素深度图  $\hat{D}_t$。 姿势网络将目标视图($I_t$)和临近帧/源视图(例如, $I_{t-1}$和$I_{t+1}$)作为输入, 并输出相对相机姿势$\hat{T}_{t\rightarrow t-1}, \hat{T}_{t\rightarrow t+1} $). 然后使用两个网络的输出来对源视图进行反向求解以重建目标视图, 并且几何一致性重建损失用于训练CNN. 通过利用视图合成作为监督信号, 我们能够以无监督的方式从视频中训练整个框架. }
		\label{fig:sfmlearner}%label放在caption下面貌似才得行, 要不然有问号
	\end{center}
\end{figure}


该文为当年CVPR oral, 结果虽比有监督差, 但也相当具有竞争力. 不过同时也具有以下缺陷 a) 该方法并不能很好的处理背景运动(相机运动)与景物运动的关系,遮挡问题也会造成一定的影响 b)该方法用到了内参矩阵作为几何一致性损失项的重要参数, 而且假定内参矩阵已经给出, 也就是说虽然该方法可以仅通过单目图像序列做训练, 但必须要有与之匹配的内参矩阵, 这使得整体框架有了一定的局限性, 另外值的一提的是, 同为当年oral的\cite{monodepth}为双目无监督训练, 取得了kitti中深度估计任务最好的结果, 这也是首次无监督训练超过\cite{Eigen2014}的有监督深度估计在kitti中的结果.


%monodepth
\cite{Godard2019monodepth2}也是使用类似架构, 通过DepthNet 与 PoseNet二者结合, 通过几何一致性损失来对两个网络做优化. 该方法比较之前的工作,主要贡献在于三点, 第一点, 改进了损失函数, 提供了一种对待遮挡问题鲁棒性更高的方法. 这里的遮挡问题与其他任务有些不同, 主要有两种遮挡,a) 是由于相机自运动造成的遮挡,比如有些像素在t时刻可以观察到再t+1时刻就观察不到;b) 指的是由于相机运动在t时刻图像的边缘部分会在t+1时刻消失在视野内的情况.第二个贡献在于通过改进损失函数的办法较好的解决了深度估计中常出现的深洞效应(depth holes),这种情况是由于前方景物长时间与相机相对运动近乎为零, 导致将其误认为背景而误判深度为无限远而造成的.第三个贡献为采全分辨率多尺度采样方法,减少了视觉伪影.

整体来说该方法旨在解决单目估计的常见问题, 提升估计的效果, 并在kitti深度估计任务中也取得了最好结果.