%transfer_learning

\section{迁移学习}

深度估计任务中, 迁移学习在从无监督方法开始后就一直在用, 这是由于大部分无监督方法属于一种由粗到细(coarse-to-fine)的方法. 具体来讲就是在序列图像传入网络后会总会现将某一目标帧$I_t$先传入一个预训练的子网络DepthNet(为了描述方便, 都使用该称呼)得到一个较粗的深度估计结果$D_t$, 后面在通过其他帧的几何约束不断的优化网络, 使得最后生成的深度图$D_t$与其他任务的特征进行反向求解, 得到一个$I_t$, 如果两者度量相近, 则说明效果理想.

明显可见, 如果预训练子网络DepthNet得到的粗结果越接近最终结果, 那么整体网络的训练会收敛越快, 所以预训练子网络的性能极大决定了整体框架的性能, 而且DepthNet的训练数据集也能极大影响整体的泛化性能. 在研究初期, 无监督多采用了\cite{Eigen2014}中的网络作为骨干网络, 该网络在三个具有ground-truth的数据集kitti\cite{kitti2012}, CityScape\cite{cityscape}和NYUdepth\cite{nyudepth}进行训练, 数据集场景主要在室外道路或者室内,另外, \cite{MDLi18}在将其作为预训练网络后, 还将应用场景进行了场景泛化测试, 得到结果如\ref{fig:megadepth}所示, 效果比较理想, 但是值的注意的是, 测试的两张图片除了尺度和景物信息, 在景物分布和视锥地面夹角上与训练数据集其实差别不大(大楼林立的场景在景物分布上与室内场景非常类似). 这也是直观效果较好的原因之一. 另外值得一提的是\cite{Eigen2015}还讨论了数据集的训练顺序不同而造成的影响问题. 



\begin{figure*}[t]
	\centering        
	\includegraphics[width=0.8\linewidth]{other/megadepth.jpg}
	\caption{数据集经过KITTI和CityScape训练后, 可以较好的泛化在一些类似场景中, 比如第一列和街道场景类似, 第二,四列与室内场景在景物排列上相似}
	\label{fig:megadepth}
\end{figure*}

\cite{Gordon2019}将无监督深度估计的测试场景首次泛化到航拍序列中, 值的注意的是其用的训练数据集还包括了EuRoC\cite{Burri25012016}, 这是一个通过无人机在工厂厂房里采集到的具有ground-truth的数据集, 目的也是通过工厂厂房能更好的模拟室外城市楼群. 测试场景来自数据集YouTube8M\cite{youtube8m}

为了更好的将在自动驾驶领域数据集的单目训练方法迁移到航拍中, 预训练网络至关重要, 在航拍数据集中获得较为准确的ground-truth并不现实, 所以, 通过除了尺度信息不同而其他场景类似的, 带有ground-truth的微缩数据集进行预训练,并将其作为子网络, 在相同场景的无标签数据集上, 联合其他任务一起做无监督训练, 这不失为一种更好的办法.



这些问题是属于一个小方向称为域适应问题(Domain Adaption), 即:如何仅通过数据集A(通常为带有Ground-truth的虚拟数据集)的有监督训练来使得模型在数据集B(一般为Ground-trut获取难度较大的真实数据集)上表现较好。 上述是一种方法:通过数据集A的预训练再使用B进行无监督训练, 还有一种方法:找到数据集B中$X_r$到虚拟数据集$X_s$的映射\cite{Atapour-Abarghouei2018}