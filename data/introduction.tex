\section{引言}

在机场中,塔台管制员通过目视的方法观察机场场面中航空器、车辆等目标的运行状态以及是否存在非合作目标和鸟群,获取航空器与车辆的位置、起飞降落时飞机的位姿、飞机的机尾号、机型、所属航司等信息,判断是否存在场面冲突的威胁,并进行合理的指挥控制。随着中国航空运输的快速发展,机场运输量和保障架次大幅提升,机场规模不断扩大,机场地面交通流量越来越大,人、设备等场面运行要素不断增加,导致机场运输愈加复杂,现行的目视管制方法无法感知全区域、全天候的机场运行态势。而基于视频的机场场面监视为目视管制方法的缺陷提供了有效的解决方案,实现对机场场面全方位的监控。但现有的机场场面监视方法缺少智能化,无法像管制员那样全方面获取机场中目标的运行态势,如:只能获取目标的位置。利用基于视频的监视手段辅助管制员对机场进行管理,则需要从视频中获取目标最全面的信息以及对未来运行态势的估计,如:目标的位置、速度、姿态、是否存在非合作目标以及鸟群、是否存在运行冲突,即:实现全方面的、精准的航空运行态势感知为管制员提供便利。多任务学习为智能化机场场面监视提供了有效的解决途径,通过对监控视频的分析,获取场面中目标的所有信息。而这些信息存在的内在联系也为基于机场场面安全的多任务学习提供了可能性。


人是一种能同时进行学习多种知识,也能同时执行多种任务的生物。在进行多种知识学习的过程中,人能通过利用之前积累的知识来帮助待掌握的知识的学习,以提升效率,同时还能保证知识间的互补干扰。在执行多种任务的时候,能有效合理规划任务的安排,根据任务的不同,分配适当的时间和精力。而这种学习以及执行的方式就是未来智能化的发展方向。多任务学习(MTL)就是这样一个智能系统的雏形,Caruana在上世纪对多任务学习的进行了一个简单的描述“leveraging the domain-specific information contained in the training signals of related tasks”\cite{caruana1997multitask}。多任务学习认为:不同的任务之间是存在关联的,具有强耦合性的任务之间的关联性更强,而这种关联性就确定了这些任务之间的信息存在某种潜在联系,能相互补充,从而提升所有任务的泛化能力。多任务学习具有以下几个优点:1、解决训练数据稀疏的问题;2、利用已有的知识来解决未知的问题;3、通过信息、特征共享,提升特定任务的效率以及精度;4、提升所有任务的泛化能力;5、系统的算力限制导致无法多个单任务系统并行运算。可以解决由于航空运行数据采集不足导致的全面态势感知不充分以及目标定位与航空器姿态估计不准等重要问题,因此多任务学习为保障机场场面安全的未来的智能化机场场面监视方法提供了有效的解决方案。

基于机场场面安全的多任务学习仍然存在两个挑战:1)为了实现全面的航空运行态势感知,则需要实现不同任务之间的信息交互以推测未测信息;2)为了实现准确的航空运行态势感知,则需要提升所有任务的精度。这体现了全面性以及精准性,而它们严重依赖于不同任务之间的信息共享,因此,如何实现不同任务之间高效的信息交互,是基于机场场面安全的多任务学习的关键问题之一。同时,用于冲突预测的全面的、精准的数据要求所有任务的平衡,即所有任务的重要性是相同的,因此,如何实现任务之间的平衡,也是关键问题之一。基于机场场面安全的多任务学习可与其他学习方法联合起来,如迁移学习\cite{pan2009survey}、终生学习\cite{parisi2019continual}、增强学习\cite{hessel2019multi}。虽然多任务学习与迁移学习、终身学习等存在一定的相似之处,但是他们的目的是不同的。多任务学习强调的是多个任务整体性能的提升,迁移学习强调的是利用源域的知识提升目标域的性能,终生学习强调在新知识的学习时对已有知识的保留。并且三者在学习方式上也有同步与异步之分,即任务的学习是否同时进行。即便如此,多任务学习、迁移学习、终生学习甚至增强学习的训练方式是可相互借鉴的。因此,利用不同的训练方式对基于机场场面安全的多任务学习进行训练也是关键问题之一。

这篇文章从四个角度对基于机场场面安全的多任务学习进行一次综述报道,首先本文对机场场面的多任务进行分类,并对场景检测类任务集进行了任务关系的层次梳理,并提出了任务关系图。其次,对目前常用的深度神经网络参数共享机制中的典型模型进行梳理,进而分析这些模型的主要优点和缺点,最后针对上述多任务深度神经网络模型存在的问题提出一些改进策略。然后,分析基于损失函数的多任务平衡方法,总结其中的难点以及一些未来可行的研究。最后,本文重点梳理其与其他学习范式的联系以及相关交叉。
