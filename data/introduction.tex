%introduction

\section{引言}

随着深度学习的发展, 计算机视觉领域内大部分任务中的最好结果是由结合深度学习的方法来取得的. 由于结合深度学习方法的个别视觉任务中, 带标签数据的成本极高, 这促使某些任务的无监督学习成为研究热点.近年来在无监督单目训练深度估计上, 即仅通过单目图像序列能推断出场景深度的方法,取得的一系列进展, 由于该任务的病态性, 仅通过深度信息训练的网络无法取得满意的结果, 所以在\cite{Eigen2014}之后, 涌现出其他用更多约束的方法, 来减少任务估计中的误差. 比较典型的比如有\cite{Zhou2017}通过联系运动估计, 来增强深度估计任务的性能, 

由于研究应用主要在航空器视觉上, 相对大视场范围, 小视距的多目采样意义不大, 所以本文重要讨论单目训练, 有些通过多目几何约束进行多目无监督训练的方法本文暂不讨论. 另外由于历史原因, 无监督视频序列训练在有些文献中称为自监督学习, 本文不加区分二者.

本文主要讨论了近年来在无监督单目训练深度估计上的一系列工作进展.并根据发展趋势, 将章节分为多任务学习理论,引入运动估计的方法, 引入光流估计的方法, 引入分割的方法几个部分进行归纳, 最后在\ref{sec:discussion}中阐述了在远景深度估计中, 也就是在航空器视觉中, 如何抉择更好的解决此任务.



